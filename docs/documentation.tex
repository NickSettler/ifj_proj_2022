% Author: Nikita Moiseev <xmoise01@stud.fit.vutbr.cz>
% Author: Maksim Kalutski <xkalut00@stud.fit.vutbr.cz>
% Author: Elena Marochkina <xmaroc00@stud.fit.vutbr.cz>
% Author: Nikita Pasynkov <xpasyn0034@stud.fit.vutbr.cz>


\documentclass[a4paper, 11pt]{article}


\usepackage[czech]{babel}
\usepackage[utf8]{inputenc}
\usepackage{geometry} \geometry{verbose,a4paper,tmargin=2cm,bmargin=2cm,lmargin=2.5cm,rmargin=1.5cm}
\usepackage{times}
\usepackage{verbatim}
\usepackage{enumitem}
\usepackage{graphicx} % insert picture
\usepackage[unicode]{hyperref}
\hypersetup{
	colorlinks = true,
	hypertexnames = false,
	citecolor = blue
}


\begin{document}
	%% Title page %%
	\begin{titlepage}
		\begin{center}
			\includegraphics[width=0.77\linewidth]{fit_logo.png} \\

			\vspace{\stretch{0.382}}

			\Huge{Projektová dokumentace} \\
			\LARGE{\textbf{Implementace překladače imperativního jazyka IFJ22}} \\
			\Large{Tým \textbf{xmoise01}, varianta \textbf{BVS}}
			\vspace{\stretch{0.7}}
		\end{center}

		\begin{minipage}{0.4 \textwidth}
			{\Large \today}
		\end{minipage}
		\hfill
		\begin{minipage}[r]{0.6 \textwidth}
			\Large
			\begin{tabular}{l l l}
				\textbf{Nikita Moiseev} & \textbf{(xmoise01)} & \quad 25\,\% \\
				Maksim Kalutski & (xkalut00) & \quad 25\,\% \\
				Elena Marochkina & (xmaroc00) & \quad 25\,\% \\
				Nikita Pasynkov & (xpasyn00) & \quad 25\,\% \\
			\end{tabular}
		\end{minipage}
	\end{titlepage}



	% Obsah %
	\pagenumbering{roman}
	\setcounter{page}{1}
	\tableofcontents
	\clearpage



	% Úvod %
	\pagenumbering{arabic}
	\setcounter{page}{1}

	\section{Úvod}
	Cílem projektu bylo vytvořit program v~jazyce~C, který načte zdrojový kód zapsaný ve zdrojovém jazyce IFJ22,
	jenž je zjednodušenou podmnožinou jazyka PHP a~přeloží jej do cílového jazyka IFJcode22 (mezikód).

	Program funguje jako konzolová aplikace, které načítá zdrojový program ze standardního vstupu a~generuje
	výsledný mezikód na standardní výstup nebo v~případě chyby vrací odpovídající chybový kód.



	% Návrh a implementace %
	\section{Návrh a~implementace}

	Projekt jsme sestavili z~několika námi implementovaných dílčích částí, které jsou představeny v~této kapitole.
	Je zde také uvedeno, jakým způsobem spolu jednotlivé dílčí části spolupracují.


	\subsection{Lexikální analýza}

	Při tvorbě překladače jsme začali implementací lexikální analýzy. Hlavní funkce této analýzy je \texttt{get\_next\_token},
	pomocí níž se čte znak po znaku ze zdrojového souboru a~převádí na strukturu \texttt{token}, která se skládá z~typu a~hodnoty.
	Typy tokenu jsou \texttt{EOF}, speciální znaky, speciální závorky PHP, identifikátory, klíčová slova, datové typy a~také aritmetické, relační a logické operátory a operátor přiřazení a~ostatní znaky, které mohou být použity v~jazyce IFJ2022.
	Hodnota atributu je
	\texttt{value}. Pokud je typ tokenu identifikátor, pak bude atribut daný identifikátor,
	když by byl typ tokenu klíčové slovo, přiřadí atributu dané klíčové slovo, pokud číslo, atribut bude ono číslo. S~takto vytvořeným tokenem poté pracují další analýzy.

	Celý lexikální analyzátor je implementován jako deterministický konečný automat podle předem vytvořeného diagramu
	\ref{figure:fa_graph}. Konečný automat je v~jazyce C~jako jeden nekonečně opakující se \texttt{switch}, kde každý případ
	\texttt{case} je ekvivalentní k~jednomu stavu automatu. Pokud načtený znak nesouhlasí s~žádným znakem, který jazyk povoluje,
	program je ukončen a~vrací chybu 1 \texttt{LEXICAL ERROR CODE 1}. Jinak se přechází do dalších stavů a~načítají se další znaky, dokud nemáme hotový jeden
	token, který potom vracíme a~ukončíme tuto funkci.


	\subsection{Syntaktická analýza}

	Nejdůležitější částí celého programu je syntaktická analýza.

	Až na výrazy se syntaktická analýza řídí LL -- gramatikou a~metodou rekurzivního sestupu podle pravidel v~LL - tabulce
	\ref{table:ll_gramatika}.


	\subsection{Sémantická analýza}




	\subsection{Generování cílového kódu}


    % Práce v týmu %
	\section{Práce v~týmu}

	\subsection{Způsob práce v~týmu}

	Na projektu jsme začali pracovat na začátku října. Práci jsme si dělili postupně, tj. neměli jsme od začátku
	stanovený kompletní plán rozdělení práce. Na dílčích částech projektu pracovali většinou
	dvojice členů týmu.

	\subsubsection{Verzovací systém}

	Pro správu souborů projektu jsme používali verzovací systém Git. Jako vzdálený repositář jsme používali \mbox{GitHub}.

	Git nám umožnil pracovat na více úkolech na projektu současně v~tzv. větvích. Většinu úkolů jsme nejdříve připravili
	do větve a~až po otestování a~schválení úprav ostatními členy týmu jsme tyto úpravy začlenili do hlavní
	vývojové větve.

	\subsubsection{Komunikace}

	Komunikace mezi členy týmů probíhala převážně osobně nebo prostřednictvím aplikace Telegram.

	V~průběhu řešení projektu jsme měli i~osobní setkání každý týden, kde jsme probírali a~řešili problémy týkající se různých částí projektu. Pro plánování úkolů jsme použili webovou aplikaci Notion.

	\subsection{Rozdělení práce mezi členy týmu}

	Práci na projektu jsme si rozdělili rovnoměrně s~ohledem na její složitost a~časovou náročnost.
	Každý tedy dostal procentuální hodnocení 25\,\%.
	Tabulka \ref{table:rozdeleni_prace} shrnuje rozdělení práce v~týmu mezi jednotlivými členy.
	\bigskip
	\begin{table}[ht]
		\centering
		\begin{tabular}{| l | l |}
			\hline
			Člen týmu & Přidělená práce \\ \hline
			\textbf{Nikita Moiseev} & \begin{tabular}{l} vedení týmu, organizace práce, dohlížení na provádění práce,
				konzultace, \\ kontrola, lexikální analýza, syntaktická analýza \end{tabular} \\
			Maksim Kalutski & \begin{tabular}{l} generování cílového kódu, testování \end{tabular} \\
			Elena Marochkina & \begin{tabular}{l} implementace tabulky symbolů, syntaktická analýza, testování, dokumentace \end{tabular} \\
			Nikita Pasynkov & \begin{tabular}{l} lexikální analýza, syntaktická analýza, testování, dokumentace \end{tabular} \\ \hline
		\end{tabular}
		\caption{Rozdělení práce v~týmu mezi jednotlivými členy}
		\label{table:rozdeleni_prace}
	\end{table}



	%Závěr %
	\section{Závěr}




	%Citace %
	\clearpage
	\bibliographystyle{czechiso}
	\renewcommand{\refname}{Literatura}
	\bibliography{dokumentace}



	%Přílohy%
	\clearpage
	\appendix


	\section{Diagram konečného automatu specifikující lexikální analyzátor}
	\begin{figure}[!ht]
		\centering
		\includegraphics[width=0.95\linewidth]{fsm.png}
		\caption{Diagram konečného automatu specifikující lexikální analyzátor}
		\label{figure:fa_graph}
	\end{figure}

    \newpage
	\section{LL -- gramatika}
	    \begin{table}[!ht]
		\centering
		\begin{enumerate}[noitemsep]
			\item \verb|<prog> -> <?php <declare> <f-dec-stats> <stat-list> ?>|
			\item \verb|<prog> -> <? <declare> <f-dec-stats> <stat-list> ?>|

			\item \verb|<f-dec-stats> -> <f-dec-stat>|
			\item \verb|<f-dec-stats> -> <f-dec-stat> <f-dec-stats>|
			\item \verb|<f-dec-stat> -> function ID ( <f-args> ) : <f-type> { <stat-list> }|
			\item \verb|<f-dec-stat> -> function ID ( <f-args> ) { <stat-list> }|

			\item \verb|<f-type> -> int|
			\item \verb|<f-type> -> float|
			\item \verb|<f-type> -> string|
			\item \verb|<f-type> -> void|

			\item \verb|<f-args> -> <f-arg>|
			\item \verb|<f-args> -> <f-arg>, <f-args>|
			\item \verb|<f-args> -> |$\varepsilon$
			\item \verb|<f-arg> -> <f-type> ID|

			\item \verb|<stat> -> ID = <expr> ;|
			\item \verb|<stat> -> ID = ID ( <args> ) ;|
			\item \verb|<stat> -> ID ( <args> ) ;|
			\item \verb|<stat> -> return <expr> ;|
			\item \verb|<stat> -> if ( <expr> )|
			\item \verb|<stat> -> if ( <expr> ) <stat> else <stat>|
			\item \verb|<stat> -> while ( <expr> ) <stat>|
			\item \verb|<stat> -> { <stat-list> }|

			\item \verb|<args> -> <arg>|
			\item \verb|<args> -> <arg>, <args>|
			\item \verb|<arg> -> <term>|
			\item \verb|<arg> -> |$\varepsilon$

			\item \verb|<term> -> int|
			\item \verb|<term> -> float|
			\item \verb|<term> -> string|
			\item \verb|<term> -> NULL|
			\item \verb|<term> -> ID|

			\item \verb|<stat-list> -> <stat> <stat-list>|
			\item \verb|<stat-list> -> |$\varepsilon$

			\item \verb|<expr> -> EXPR <expr>|
			\item \verb|<expr> -> ID ( <args> )|
			\item \verb|<expr> -> |$\varepsilon$

			\item \verb|<declare> -> declare(strict-types = 0) ;|
			\item \verb|<declare> -> declare(strict-types = 1) ;|

		\end{enumerate}

		\caption{LL -- gramatika řídící syntaktickou analýzu}
		\label{table:ll_gramatika}
	\end{table}

    \newpage
	\section{LL -- tabulka}
	\begin{table}[!ht]
		\centering
		\includegraphics[width=1\linewidth]{}
		\caption{LL -- tabulka použitá při syntaktické analýze}
		\label{table:ll_table}
	\end{table}

    \newpage
	\section{Precedenční tabulka}
	\begin{table}[!ht]
		\centering
		\includegraphics[width=0.7\linewidth]{precedence_table.png}
		\caption{Precedenční tabulka použitá při precedenční syntaktické analýze výrazů}
		\label{table:prec_table}
	\end{table}


\end{document}
